\documentclass[12pt,a4paper]{article}           
\usepackage{gvv}        
\graphicspath{{/storage/emulated/0/latex/image.jpg}}

\begin{document}
\title{\underline{\textbf{2023-2}}}      \date{}\maketitle                    

\begin{enumerate}
\item Let $f :[1, \infty) \to \mathbb{R}$ be a differentiable function such that $f(1) = \frac{1}{3}$ and $3 \int_{1}^{x}  f(t) dt = x f(x) - \frac{x^2}{3},x \in [1, \infty).$ Let $e$ denote the base of the natural logarithm.Then the value of $f(e)$ is\\
\begin{enumerate}
\item $\frac{e^2+4}{3}$
\item $\frac{\log_{e}{4} + e}{3}$
\item $\frac{4e^2}{3}$
\item $\frac{e^2-4}{3}$
\end{enumerate}

\item Consider an experiment of tossing a coin repeatedly until the outcomes of two consecutive tosses are the same. If the probability of a random toss resulting in heads is $\frac{1}{3}$, then the probability that the experiment stops with head is\\
\begin{enumerate}
\item $\frac{1}{3}$                             \item $\frac{5}{21}$                            \item $\frac{4}{21}$
\item $\frac{2}{7}$
\end{enumerate}

\item For any $y \in \mathbb{R}$, let $\cot^{-1}(y) \in (0, \pi) \text{and} \tan^{-1}(y) \in \brak( -\frac{\pi}{2}, \frac{\pi}{2} \brak)$ .Then the sum of all the solutions of the equation $\tan^{-1}(\frac{6y}{9-y^2}) + \cot^{-1}(\frac{9-y^2}{6y}) =\frac{2\pi}{3}$ for $0 < |y| < 3$ is equal to\\
\begin{enumerate}
\item $2\sqrt{3}-3$
\item $3 - 2\sqrt{3}$
\item $4\sqrt{3} - 6$
\item $6-4\sqrt{3}$
\end{enumerate}


\item Let the position vectors of the points $P, Q, R,$ and $S$ be
$\vec{\overrightarrow{a}} = \hat{i} + 2\hat{j} - 5\hat{k}$, $\vec{\overrightarrow{b}} = 3\hat{i} + 6\hat{j} + 3\hat{k}$, $\vec{\overrightarrow{c}} = \frac{17}{5}\hat{i} + \frac{16}{5}\hat{j} + \frac{7}{5}\hat{k}$, and $\vec{\overrightarrow{d}} = 2\hat{i} + \hat{j} + \hat{k}$, respectively. Then which of the following statements is true?
\begin{enumerate}
\item The points $P, Q, R,$ and $S$ are NOT coplanar
\item $\frac{\vec{\overrightarrow{b}} + 2\vec{\overrightarrow{d}}}{3}$ is the position vector of a point which divides $PR$ internally in the ratio $5:4$
\item $\frac{\vec{\overrightarrow{b}} + 2\vec{\overrightarrow{d}}}{3}$ is the position vector of a point which divides $PR$ externally in the ratio $5:4$
\item The square of the magnitude of the vector $\vec{\overrightarrow{b}} \times \vec{\overrightarrow{d}}$ is $95$
\end{enumerate}


\item Let $M = \brak{ a_{ij} }, \, a_{ij} \in \{1,2,3\}$ be the $3 \times 3$ matrix such that $a_{ij} = 1$ if $j + 1$ is divisible by $i$, otherwise $a_{ij} = 0$. Then which of the following statements is(are) true?  
\begin{enumerate}  
\item $M$ is invertible  
\item There exists a nonzero column matrix $\myvec{a_1 \\ a_2 \\ a_3}$ such that $M \myvec{a_1 \\ a_2 \\ a_3} = \myvec{-a_1 \\ -a_2 \\ -a_3}$  
\item The set $\{ X \in \mathbb{R}^3 : MX = 0 \} \neq \{0\}$, where \quad $0= \myvec{0 \\ 0 \\ 0}$  
\item The matrix $(M - 2I)$ is invertible, where $I$ is the $3 \times 3$ identity matrix  
\end{enumerate}


\item Let $f : (0,1) \to \mathbb{R}$ be the function defined as $ f(x) =  [4x] \brak{x-\frac{1}{4}}^2 \brak{x-\frac{1}{2}},$ where $[x]$ denotes the greatest integer less than or equal to$x.$Then, which of the following statements is(are) true?
\begin{enumerate}
\item The function $f$ is discontinuous exactly at one point in $(0,1)$.
\item There is exactly one point in $(0,1)$ at which the function $f$ is continuous but NOT differentiable.
\item The function $f$ is NOT differentiable at more than three points in $(0,1)$.
\item The minimum value of the function $f$ is $\frac{1}{512}$.
\end{enumerate}

\item Let $S$ be the set of all twice differentiable functions $f from \mathbb{R} to \mathbb{R}$ such that $ \frac{d^2 f}{dx^2}(x) > 0  \text{for all}  x \in (-1, 1).$ For $f \in S$, let ${X_{f}}$ be the number of points $x \in (-1, 1)$ where $f(x) = x$. Then, which of the following statements is(are) true?
\begin{enumerate}
\item There exists a function $f \in S$ such that ${X_{f}} = 0$.
\item For every function $f \in S$, we have ${X_{f}} \leq 2$.
\item There exists a function $f \in S$ such that ${X_{f}} = 2$.
\item There does NOT exist any function $f \quad in \quad S$ such that ${X_{f}} = 1$.
\end{enumerate}

\item For $x \in \mathbb{R}$, let $\tan^{-1}(x) \in \brak( -\frac{\pi}{2}, \frac{\pi}{2} \brak).$ Then the minimum value of the function $f: \mathbb{R} \to \mathbb{R}$ defined by $f(x) =  \int_0^{x\tan^{-1}x} \frac{e^{t-\cos t}}{1+t^{2023}}\, dt$

\item For $x \in \mathbb{R}$, let $y(x)$ be a solution of the differential equation $ (x^2 - 5) \frac{dy}{dx} - 2xy = 2x{(x^2 - 5)^2}$  such that $y(2) = 7$. Then the maximum value of thefunction $y(x)$ is

\item Let $X$ be the set of all five-digit numbers formed using the digits 1, 2, 2, 2, 4, 4, 0. For example, 22240 is in $X$, while 02244 and 44422 are not in $X$. Suppose that each element of$X$ has an equal chance of being chosen. Let $p$be the conditional probability that an element chosen at random is a multiple of 20 given that it is a multiple of 5. Then the value of $38p$ isequal to

\item Let $A_1, A_2, A_3, \dots, A_8$ be the vertices of a regular octagon that lie on a circle of radius 2. Let $P$ be a point on the circle, and let $PA_i$ denote the distance between the points $P$ and $A_i$ for $i = 1,2, \dots, 8$. If $P$ varies over the circle, then the maximum value of the product $PA_1 \cdot PA_2 \cdot PA_3 \cdots PA_8$is

\item Let $R = \brak{ \begin{bmatrix} a & b \\ c & d \end{bmatrix} : a, b, c, d \in \{0,3,5,7,11,13,17,19\} }.$
Then the number of invertible matrices in $R$ is


\item Let $C_1$ be the circle of radius 1 with center at the origin. Let $C_2$ be the circle of radius $r$ with center at the point $A = (4,1)$,where $1 < r < 3$. Two distinct common tangents $PQ$ and $ST$ of $C_1$ and $C_2$ are drawn. The tangent $PQ$ touches $C_1$ at $P$ and $C_2$ at $Q$. The tangent $ST$ touches $C_1$ at $S$ and $C_2$ at $T$. Midpoints of the line segments $PQ$ and $ST$ are joined to form a line which meets the $x$-axis at a point $B$. If $AB = \sqrt{5}$, then the value of $r^2$ is



\noindent Consider an obtuse angled triangle $ABC$ in which the difference between the largest and the smallest angle is $\frac{\pi}{2}$ and whose sides are in arithmetic progression. Suppose that the vertices of this triangle lie on a circle of radius 1.

\item Let $a$ be the area of the triangle $ABC$. Then the value of $(64a)^2$ is

\item Then the in radius of the triangle ABC is


\noindent Consider the $6 \times 6$ square in the figure. Let $A_1, A_2, \ldots, A_{49}$ be the points of intersections (dots in the picture) insome order. We say that $A_i$ and $A_j$ are friends if they are adjacent along a row or along a column. Assume that each point $A_i$ has an equal chance of being chosen


\begin{figure}[h]
    \centering
	\includegraphics[width=\columnwidth]{image.jpg}
    \caption{A 6x6 Square}
    \label{fig:Square}
\end{figure}


\item Let $p_i$ be the probability that a randomly chosen point has $i$ many friends, $i = 0, 1, 2, 3, 4$. Let $X$ be a random variable such that for $i = 0, 1, 2, 3, 4$, the probability $P(X = i) = p_i$. Then the value of $7E(X)$ is


\item Two distinct points are chosen randomly out of the points $A_1, A_2, A_3, \dots, A_{49}$.
Let $p$ be the probability that they arefriends.Then the value of $7p$ is

\end{enumerate}
\end{document}
